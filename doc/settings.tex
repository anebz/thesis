% Template for a Thesis
%
% settings.tex
%
% Here general setting are made

\documentclass[%
	a4paper,%							A4 paper
	oneside,%							oneside (left and right margin are equal
	bibliography=totoc,%	add the bibliography to the table of contents
	%listof=totoc,%				add the list of figures and list of tables to the table of contents
	numbers=noenddot,%		no dot at the end of heading numbers (see the comment below)
	headsepline,%					line after the page head
	footsepline,%					line before the page foot
	headings=small,%			smaller headings
	12pt,%								font size
]{scrreprt}

% According to the famous German dictionary 'Duden'
% (see https://en.wikipedia.org/wiki/Duden), the following rules apply:
% If only Arabic numbers are used to number the sections, no end dot is used.
% If also Roman numbers or letters are used to number the sections, a terminatory dot shall be placed.
% The KoMA script tries to determine the used numbering 
% scheme in its default setting and applies the corresponding rule.
% (This information is saved in the .aux file,
% therefore a second LaTeX run might be necessary.)

% package for english language
\usepackage[english]{babel}

% input encoding
\usepackage[utf8]{inputenc}

% output font encoding
\usepackage[T1]{fontenc}

% output font
\usepackage{lmodern}

% microtypographical improvements
\usepackage{microtype}

% more options for tables
\usepackage{array}

% nicer tables
\usepackage{booktabs}

% package to check for pdf mode
\usepackage{ifpdf}

% package for graphics
% the formats png, pdf, jpg are permitted
\usepackage{graphicx}
% default path for figures
\graphicspath{{figures/}}
% default file extensions for figures
\ifpdf
	% for pdf output
	\DeclareGraphicsExtensions{.pdf,.jpg,.png}
\else
	% for html output
	\DeclareGraphicsExtensions{.png,.jpg}
\fi

% package to allow for dots in filenames
\usepackage{grffile}

% package for the "drawing" of figures
\usepackage{tikz}

% package for rendering plots using TikZ
\usepackage{pgfplots}
% always use the newest version
\pgfplotsset{compat=newest}
% format and size template for two plots side-by-side
\pgfplotsset{
	scriptsize/.style={
		width=4.5cm,
		height=,
		legend style={font=\scriptsize},
		tick label style={font=\scriptsize},
		label style={font=\footnotesize},
		title style={font=\footnotesize},
		every axis title shift=0pt,
		max space between ticks=15,
		every mark/.append style={mark size=7},
		major tick length=0.1cm,
		minor tick length=0.066cm,
	},
}
% new format setting for a usual coordinate system
\pgfplotsset{
	coordinatesystem/.style={
		axis x line=middle,
		axis y line=center,
		xmajorgrids=false,
		ymajorgrids=false,
		legend style={draw=none},
	},
}
% align legends entries to the left
\pgfplotsset{legend cell align=left}
% draw a main grid
\pgfplotsset{xmajorgrids}
\pgfplotsset{ymajorgrids}
% number of minor ticks between two major tickmarks
%\pgfplotsset{minor x tick num={3}}
%\pgfplotsset{minor y tick num={3}}
% draw a minor grid
%\pgfplotsset{xminorgrids}
%\pgfplotsset{yminorgrids}
% scale only the axis using a certain width and height
\pgfplotsset{scale only axis}
% define colors like in MATLAB (version R2014a and earlier)
%\definecolor{matlab1}{rgb}{0,0,1}
%\definecolor{matlab2}{rgb}{0,0.5,0}
%\definecolor{matlab3}{rgb}{1,0,0}
%\definecolor{matlab4}{rgb}{0,0.75,0.75}
%\definecolor{matlab5}{rgb}{0.75,0,0.75}
%\definecolor{matlab6}{rgb}{0.75,0.75,0}
%\definecolor{matlab7}{rgb}{0.25,0.25,0.25}
% define colors like in MATLAB (since version R2014b)
% see: https://de.mathworks.com/help/matlab/graphics_transition/why-are-plot-lines-different-colors.html
% The RGB values can also be read out via:
% >> figure;
% >> get(gca,'defaultAxesColorOrder');
% Thanks to Thomas Aab for the update.
\definecolor{matlab1}{rgb}{0,0.4470,0.7410}
\definecolor{matlab2}{rgb}{0.8500,0.3250,0.0980}
\definecolor{matlab3}{rgb}{0.9290,0.6940,0.1250}
\definecolor{matlab4}{rgb}{0.4940,0.1840,0.5560}
\definecolor{matlab5}{rgb}{0.4660,0.6740,0.1880}
\definecolor{matlab6}{rgb}{0.3010,0.7450,0.9330}
\definecolor{matlab7}{rgb}{0.6350,0.0780,0.1840}
% define a color cycle list like in MATLAB
\pgfplotscreateplotcyclelist{matlab}{
	{matlab1,solid},
	{matlab2,dashed},
	{matlab3,dashdotted},
	{matlab4,dotted},
	{matlab5,densely dashed},
	{matlab6,densely dashdotted},
	{matlab7,densely dotted}%this prevents an error
}
% use the earlier defined color list
\pgfplotsset{cycle list name=matlab}
% use the standard color list of pgfplots
%\pgfplotsset{cycle list name=color list}
% use only grayscale lines
%\pgfplotsset{cycle list name=linestyles}
% use a line width of 1pt
\pgfplotsset{every axis plot/.append style={line width=1pt}}
% use a comma as the dot separator for german documents
\addto\extrasngerman{\pgfplotsset{/pgf/number format/.cd,set decimal separator={{{,}}}}}
% use a half space as the 1000 separator
\pgfplotsset{/pgf/number format/.cd,1000 sep={\,}}

% package for subfigures in one figure
\usepackage{subfig}

% mathematical symbols
\usepackage{amsmath}
\usepackage{amssymb}

% package to change the line spacing
\usepackage{setspace}
% use onehalf line spacing
\onehalfspacing
% set headlines in single spacing
% see http://www.mrunix.de/forums/showthread.php?p=356306
\addtokomafont{pageheadfoot}{\linespread{1}\selectfont}

% package to rotate text
\usepackage{rotating}

% package for colors
\usepackage{color}

% package for better quotes
\usepackage{csquotes}

% package for better citations
\usepackage{cite}

% package to not account for citations in the toc, lof or lot during the numbering
\usepackage{notoccite}

% package for underscores
\usepackage[normalem]{ulem}

% package for SI units
\usepackage[binary-units]{siunitx}
% options
% real fractions
\sisetup{per-mode=fraction,mode=math}
% decimal marker in dependence from the language
\addto\extrasngerman{\sisetup{output-decimal-marker={,}}}
\addto\extrasenglish{\sisetup{output-decimal-marker={.}}}
% range phrase in dependence from the language
\addto\extrasngerman{\sisetup{range-phrase={ bis~}}} 
\addto\extrasenglish{\sisetup{range-phrase={ to~}}}

% define new mathematical functions
\DeclareMathOperator{\arcosh}{arcosh}

% differential operator (small upright d)
\newcommand*{\diff}{\mathop{}\!\mathrm{d}}

% redefine the exponential function
\DeclareMathOperator{\e}{e}
\renewcommand{\exp}[1]{\e^{#1}}

% write indices upright (in Roman)
\newcommand{\ind}[1]{\mathrm{#1}}

% a period in a math environment
\newcommand{\period}{\,\text{.}}
% a comma in a math environment
\newcommand{\comma}{\,\text{,}}

% magnitude or absolute value
\newcommand{\abs}[1]{\left| #1 \right|}

% mean value or average value
\newcommand{\mean}[1]{\left\langle #1 \right\rangle}

% package for nice fractions in the text mode
\usepackage{xfrac}

% set page margin with the geometry package
\usepackage[margin=2.5cm]{geometry}

% set the hyphenation for certain words
\hyphenation{op-tic-al net-works semi-con-duc-tor}

% package for a notation/nomenclature
\usepackage{nomencl}
\makenomenclature

% package for a list of abbreviations
% refer to: http://www.ctan.org/tex-archive/macros/latex/contrib/acronym/acronym.pdf
\usepackage{acronym}

% always try to place figures and tables 'here'
\makeatletter
\renewcommand{\fps@figure}{htbp}
\renewcommand{\fps@table}{htbp}
\makeatother

% portion of floating figures and tables at the top of a page
% default value: 0.7
\renewcommand{\topfraction}{0.8}

% portion of floating figures and tables at the bottom of a page
% default value: 0.3
\renewcommand{\bottomfraction}{0.33}

% one of these values (\topfraction and \bottomfraction) has to be larger than \floatpagefraction

% portion of floating figures and tables for a new page only for floating objects
% default value: 0.5
\renewcommand{\floatpagefraction}{0.66}

% portion of a page for floating objects that is reserved for text
% default value: 0.2
\renewcommand{\textfraction}{0.10}

% numbering till subsection in the text
\setcounter{secnumdepth}{2}

% numbering till subsection in the table of contents
\setcounter{tocdepth}{2}

% new line after paragraph
% siehe http://tex.stackexchange.com/questions/40943/new-line-and-no-indent-after-paragraph
\makeatletter
\renewcommand\paragraph{\@startsection{paragraph}{4}{\z@}%
  {-3.25ex \@plus -1ex \@minus -0.2ex}%
  {0.01pt}%
  {\raggedsection\normalfont\sectfont\nobreak\size@paragraph}%
}
\makeatother

% package for an intelligent space character
\usepackage{xspace}

% abbrevations
\newcommand{\eg}{e.\,g.\xspace}
\newcommand{\ie}{i.\,e.\xspace}

% hyperref package for links in the document
\usepackage[
	pdftitle={Degree Thesis},
	pdfauthor={John Doe},
	pdfcreator={MiKTeX, LaTeX with hyperref and KOMA-Script},
	pdfsubject={Guidance on the Preparation of Degree Theses},
	pdfkeywords={Structure, Contents, Layout},
	pdfstartview={Fit}]{hyperref}

% end of file